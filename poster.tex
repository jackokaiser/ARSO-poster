%==============================================================================
%== template for LATEX poster =================================================
%==============================================================================
%
%--A0 beamer slide-------------------------------------------------------------
\documentclass[final]{beamer}
\usepackage[orientation=portrait,size=a0,
            scale=1.25         % font scale factor
           ]{beamerposter}

\geometry{
  hmargin=2.5cm, % little modification of margins
}

%
\usepackage[document]{ragged2e}
\usepackage[utf8]{inputenc}
\usepackage{tikz}
\usepackage{pgfplots}
\usepackage{graphicx}
\usepackage{caption}
\usepackage{subcaption}
\usetikzlibrary{positioning,calc,shapes, arrows}
\captionsetup[figure]{labelformat=empty}% redefines the caption setup of the figures environment in the beamer class.
\captionsetup[subfigure]{labelformat=empty}% redefines the caption setup of the figures environment in the beamer class.
\newcommand{\argmin}{\operatornamewithlimits{argmin}}

\definecolor{myyellow}{RGB}{255, 127, 0}
\definecolor{mypink}{RGB}{228, 0, 124}
\definecolor{ximatrixcolor}{HTML}{b7BC53}
\definecolor{svectorcolor}{HTML}{5ABAC2}
%% \definecolor{xvectorcolor}{HTML}{C34BB0}
%% \definecolor{xvectorcolor}{HTML}{3C3DB5}
\definecolor{xvectorcolor}{HTML}{7851DB}


\tikzstyle{input}=[draw,fill=myyellow,minimum width=2.6cm,thin]
\tikzstyle{method}=[draw,fill=mypink,minimum width=1.8cm]
\tikzstyle{arr}=[-latex]
\tikzstyle{dummy}=[minimum width=2.6cm]

\linespread{1.15}
%
%==The poster style============================================================
\usetheme{sharelatex}

%==Title, date and authors of the poster=======================================
\title
[ARSO, 1 - 3 July 2015, Lyon, France] % Conference
{ % Poster title
Absolute scale velocity determination combining visual and inertial measurements for drones
}

\author{ % Authors
Jacques Kaiser, Agostino Martinelli
}
\institute
[INRIA] % General University
{
  Team Chroma, INRIA
%% \inst{1} Chroma\\[0.3ex]
%% \inst{2} Other University, Neverland\\[0.3ex]
%% \inst{3} Yet Another University, Neverland
}
\date{The 3\textsuperscript{rd} of July 2015}


\begin{document}
\begin{frame}[t]
%==============================================================================
\begin{multicols}{3}
%==============================================================================
%==The poster content==========================================================
%==============================================================================

\section{Introduction}

State of the art approaches for visual-inertial sensor fusion are \structure{filter based algorithms}.
These methods are recursive by design and therefore \structure{require an initialization}.
A poor initialization can have a
dramatic impact on the performance of the estimations.
On the other hand, methods that work without initialization have been developed in computer vision but
they can only determine physical quantities \structure{up to a scale}.

Recently, a \structure{Closed-Form Solution} that recovers the initial velocity, the roll and pitch angles and the absolute scale by fusing visual and inertial measurements has been derived~\cite{Martinelli2014,Martinelli2012}.
While mathematically sound, this method is \structure{not robust} to noisy sensor data.
We added a \structure{key step} that makes the original method \structure{usable in practice}.

%% Specifically, we realized that the performance of the method is strongly affected by the gyroscope bias

%% by experimenting with real and synthetic data, we were able to improve its robustness.
%% The main contribution is the introduction of a simple method that automatically \structure{estimates the gyroscope bias} which was a major \structure{performance bottleneck}.

\begin{figure}
  \centering
  \hspace*{-1em}
  \begin{subfigure}[t]{0.3\columnwidth}
    \caption{Filter-based fusion}
    %% \resizebox{\columnwidth}{!}{
    %%   \raggedleft
    \hspace*{-3em}
    \begin{tikzpicture}[scale=0.8,every node/.style={transform shape}]
      \node (A) at (-5,0) [input, fill=blue!60] {Initial State};
      \node (dummy1) [dummy, right=2mm of A] {};
      \node (C) [input,right=2mm of dummy1] {Measurements};
      \node (D) [method,below=2cm of dummy1] {Fusion};
      \node (E) [input, below=2cm of D, fill=myyellow] {State};

      \node (dummy2) [dummy, right=2mm of E] {};
      \node (F) [input, right=2mm of dummy2] {Measurements};


      \node (G) [method, below=2cm of dummy2] {Fusion};
      \node (H) [input, below=2cm of G, fill=green] {State};

      \draw[arr] (A.south) -- (D.85);
      \draw[arr] (C.south) -- (D.95);
      \draw[arr] (D.south) -- (E.north);
      \draw[arr] (E.south) -- (G.85);
      \draw[arr] (F.south) -- (G.95);
      \draw[arr] (G.south) -- (H.north);
    \end{tikzpicture}
    %% }
  \end{subfigure}~
  \hspace*{5.5em}
  \begin{subfigure}[t]{0.3\columnwidth}
    \caption{Deterministic fusion}
    \hspace*{1.2em}
    \begin{tikzpicture}[scale=0.8,every node/.style={transform shape}]
      \node (C2) [input] {Measurements};
      \node (D2) [method,below=2cm of C2] {Fusion};
      \node (E2) [input, below=2cm of D2, fill=green] {State};

      \draw (C2.south) -- (D2.north);
      \draw (D2.south) -- (E2.north);
    \end{tikzpicture}

  \end{subfigure}
\end{figure}

\vspace{4cm}

\section{The Closed-Form Solution}

The Closed-Form Solution consists of a single equation that expresses the \textcolor{red}{initial velocity}, the \textcolor{blue}{distance to point-features} and the \textcolor{green}{gravity} linearly with respect to the visual and inertial measurements.
Note that the knowledge of the gravity in the local frame is equivalent to the knowledge of the roll and pitch angles.

\[
S_j = \textcolor{blue}{\lambda_1^i}\mu_1^i - \textcolor{red}{V} t_j - \textcolor{green}{G} \frac{t_j^2}{2} - \textcolor{blue}{\lambda^i_j} \mu^i_j
\]

This equation is valid for every observed point-feature $i$ at any time $t_j$.
We therefore have an overconstrained linear system that we can write in matrix formulation:

\[
\Xi \textcolor{xvectorcolor}{X} = S
\]

    We can therefore recover the initial velocity, the distance to the point-features and the roll and pitch angle of the drone by finding the vector $\textcolor{xvectorcolor}{X}$ that satisfies this equation.
    This method does not require any initialization.

\vfill
\columnbreak

\section{Test setup}

\begin{figure}
\centering
\includegraphics[width=\columnwidth]{images/setupTestDroneError.png}
\end{figure}

\subsection{Original performance}

\begin{figure}[h!]
  \centering
  \resizebox{0.7\columnwidth}{!}{\input{graph/CF2}}
  %% \caption{Performance of the Original Closed-Form Solution}
\end{figure}

With around 45\% of error on the speed and the distance to the features after 4 seconds of integration, the Original Closed-Form Solution does not perform very well on terrain data.

\section{Impact of bias}
We studied the impact of biased inertial sensors on the performance of the original Closed-Form Solution.
by experimenting with real and synthetic data, we realized that the performance is:
\begin{itemize}
 \item Strongly affected by the gyroscope bias;
 \item Weakly affected by the accelerometer bias.
\end{itemize}

  \begin{figure}[h!]
    \caption{Speed estimation error}
    \centering
    \begin{subfigure}[b]{0.47\columnwidth}
      \caption{Varying accelerometer bias}
      \resizebox{\columnwidth}{!}{\input{graph/accBiasSpeed}}
    \end{subfigure}
    \begin{subfigure}[b]{0.47\columnwidth}
      \caption{Varying gyroscope bias}
      \resizebox{\columnwidth}{!}{\input{graph/gyroBiasSpeed}}
    \end{subfigure}
  \end{figure}


\section{Our method}

The goal is to \structure{factor out the gyroscope bias} to improve the performance of the Closed-Form Solution.
Optimally, we would add the gyroscope bias as an unknown term in $\textcolor{xvectorcolor}{X}$.
Unfortunately we \structure{cannot express the gyroscope bias linearly} with respect to the visual-inertial measurements.

We propose a different approach: a \structure{non-linear minimization of the residual with respect to the gyroscope bias}.
In other words, we recover $\textcolor{xvectorcolor}{X}$ by minimizing the cost function:

\vspace{-1cm}

\[
cost(B) = \argmin_{\textcolor{xvectorcolor}{X}} ||\Xi \textcolor{xvectorcolor}{X} - S||^2
\]
{\small \emph{With:\\
$B$ the gyroscope bias,\\
$\Xi$ and $S$ computed with respect to $B$}}

The gyroscope bias is a 3D vector $B=[B_x,B_y,B_z]$. We can plot this cost function with respect to two components of the gyroscope bias:
\begin{figure}[h!]
  \centering
  \resizebox{\columnwidth}{!}{\input{graph/costXZ2}}
  \caption{Cost function: residual with respect to $B_x$ and $B_z$}
\end{figure}

Note that the residual is almost constant with respect to the third component of the gyroscope bias.
This symmetry is induced by the strong weight of the gravity in the system.
In order to prevent our optimization to converge towards high and unlikely values of the gyroscope bias, we add a regularization term to our cost function:

\[
cost(B) = \argmin_{\textcolor{xvectorcolor}{X}} ||\Xi \textcolor{xvectorcolor}{X} - S||^2 \textcolor{blue}{+ \lambda \times |B|}
\]

\section{Results}
\begin{figure}[h!]
  \centering
  \caption{Speed estimation error over a time sequence (duration set to 3s)}
  \resizebox{0.7\columnwidth}{!}{\input{graph/speed}}
\end{figure}

The original method estimates the speed with approximately 80\% error whereas our method can estimate it with 10\% error.
The performance of our method is comparable to the one of state-of-the art Kalman Filter approach, although our technique
does not require an initialization and now provides the gyroscope bias.



%==============================================================================
%==End of content==============================================================
%==============================================================================

%--References------------------------------------------------------------------

\subsection{References}

\bibliographystyle{plain}
\bibliography{library}

%--End of references-----------------------------------------------------------

\end{multicols}

%==============================================================================
\end{frame}
\end{document}
